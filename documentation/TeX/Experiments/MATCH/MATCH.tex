\subsubsection{Description}
This example shows the ability to use Cellular neural networks to find patterns in picture, this example shows a search for 3x3  pattern designated by B matrix. 
\\ Patter is set up as B matrix by the positions of -1/0/1, where -1 represents white, 1 black and 0 is for non specific color. 
\subsubsection{Setup}

\textbf{Input 1:} Arbitrary\\
\textbf{Initial state/Input 2:} Grayscale picture.\\
\textbf{Boundary conditions:} Arbitrary (0).\\
\textbf{Output:} Binary.\\
\textbf{Gene:} -6.5;0;0;0;0;1;0;0;0;0;1;-1;1;0;1;0;1;-1;1\\

\begin{minipage}{0.9\linewidth}
\begin{equation}
A =
\begin{bmatrix}
 0 &  0 &  0 \\
  0 &  1 &  0 \\
  0 &  0 &  0
\end{bmatrix}
B =
\begin{bmatrix}
 1 & -1 & 1 \\
 0 & 1 & 0 \\
 1 & -1 & 1
\end{bmatrix}
Z = -6.5
\end{equation}
\captionof{figure}{Chosen values of A,B and Z for this experiment}
\end{minipage}
\subsubsection{Results}
Figure \ref{fig:input-MA} show input used in this example, it is a picture with several black patterns. The Figure \ref{fig:output-MA} shows typical result of this example with a given B where black pixels marks the position of the middle pixel of recognised pattern. For different patterns B has to be adjusted. \\

\begin{minipage}{0.5\linewidth}
	\centering
	\includegraphics[width=0.9\linewidth]{./Experiments/MATCH/fig/Input.png} 
	\captionof{figure}{Input}
	\label{fig:input-MA}
\end{minipage}
\begin{minipage}{0.5\linewidth}
	\centering
	\includegraphics[width=0.9\linewidth]{./Experiments/MATCH/fig/Output_A.png}
	\captionof{figure}{Output}
	\label{fig:output-MA}
\end{minipage}
