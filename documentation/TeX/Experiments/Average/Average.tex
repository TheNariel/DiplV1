\subsubsection{Description}
This example shows the ability to use Cellular neural networks to produce a binary representation of grey-scale picture base on the average of the pixels. 
\subsubsection{Setup}
\textbf{Input 1:} Arbitrary\\
\textbf{Initial state/Input 2:} Grayscale picture.\\
\textbf{Boundary conditions:} Arbitrary (0).\\
\textbf{Output:} Binary.\\
\textbf{Gene:} 0;0;1;0;1;2;1;0;1;0;0;0;0;0;0;0;0;0;0\\

\begin{minipage}{0.9\linewidth}
\begin{equation}
A =
\begin{bmatrix}
 0 & 1 & 0 \\
 1 & 2 & 1 \\
 0 & 1 & 0
\end{bmatrix}
B =
\begin{bmatrix}
 0 & 0 & 0 \\
 0 & 0 & 0 \\
 0 & 0 & 0
\end{bmatrix}
Z = 0
\end{equation}
\captionof{figure}{Chosen values of A,B and Z for this experiment}
\end{minipage}

\subsubsection{Results}
Figure \ref{fig:input-AV} show input used in this example, it is grayscale picture. The Figure \ref{fig:output-AV} shows typical result of this example after running 5 cycles when this example became stable. \\

\begin{minipage}{0.5\linewidth}
	\centering
	\includegraphics[width=0.9\linewidth]{./Experiments/Average/fig/Input.png} 
	\captionof{figure}{Input}
	\label{fig:input-AV}
\end{minipage}
\begin{minipage}{0.5\linewidth}
	\centering
	\includegraphics[width=0.9\linewidth]{./Experiments/Average/fig/Output.png}
	\captionof{figure}{Output}
	\label{fig:output-AV}
\end{minipage}
