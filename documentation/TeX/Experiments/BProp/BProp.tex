\subsubsection{Description}
This example shows the ability to use Cellular neural networks to make black objects bigger (white smaller) on a binary picture. It uses a binary picture as an input, matrix setting shown bellow and fixed value boundary condition to produce a binary picture with the edges.
\subsubsection{Setup}

\textbf{Input:} Unimportant (all zeros)\\
\textbf{Boundary conditions:} Fixed.\\
\textbf{Initial output:} Grey-scale picture

\begin{minipage}{0.9\linewidth}
\begin{equation}
A =
\begin{bmatrix}
 0.25 &  0.25 &  0.25 \\
  0.25 &  3 &  0.25 \\
  0.25 &  0.25 &  0.25
\end{bmatrix}
B =
\begin{bmatrix}
 0 & 0 & 0 \\
 0 & 0 & 0 \\
 0 & 0 & 0
\end{bmatrix}
Z = 3.75
\end{equation}
\captionof{figure}{Chosen values of A,B and Z for this experiment}
\end{minipage}

\subsubsection{Results}


\begin{minipage}{0.5\linewidth}
	\centering
	\includegraphics[width=0.9\linewidth]{./Experiments/BProp/fig/Input.png} 
	\captionof{figure}{Input}
\end{minipage}
\begin{minipage}{0.5\linewidth}
	\centering
	\includegraphics[width=0.9\linewidth]{./Experiments/BProp/fig/Output.png}
	\captionof{figure}{Output}
\end{minipage}
