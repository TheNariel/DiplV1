\subsubsection{Description}
This example shows the ability to use Cellular neural networks to make black objects bigger (white smaller) on a binary picture. 
\subsubsection{Setup}

\textbf{Input 1:} Arbitrary.\\
\textbf{Initial state/Input 2:} Grayscale picture\\
\textbf{Boundary conditions:} Flux.\\
\textbf{Output:} Binary.\\
\textbf{Gene:} 3.75;0.25;0.25;0.25;0.25;3;0.25;0.25;0.25;0.25;0;0;0;0;0;0;0;0;0\\

\begin{minipage}{0.9\linewidth}
\begin{equation}
A =
\begin{bmatrix}
 0.25 &  0.25 &  0.25 \\
  0.25 &  3 &  0.25 \\
  0.25 &  0.25 &  0.25
\end{bmatrix}
B =
\begin{bmatrix}
 0 & 0 & 0 \\
 0 & 0 & 0 \\
 0 & 0 & 0
\end{bmatrix}
Z = 3.75
\end{equation}
\captionof{figure}{Chosen values of A,B and Z for this experiment}
\end{minipage}

\subsubsection{Results}
Figure \ref{fig:input-BP} show input used in this example, it is a picture with several black objects. The Figure \ref{fig:output-BP} shows typical result of this example after running 5 cycles, each cycle makes the black object grow a new layer of black pixels around them. \\

\begin{minipage}{0.5\linewidth}
	\centering
	\includegraphics[width=0.9\linewidth]{./Experiments/BProp/fig/Input.png} 
	\captionof{figure}{Input}	
	\label{fig:input-BP}
\end{minipage}
\begin{minipage}{0.5\linewidth}
	\centering
	\includegraphics[width=0.9\linewidth]{./Experiments/BProp/fig/Output.png}
	\captionof{figure}{Output}	
	\label{fig:output-BP}
\end{minipage}
